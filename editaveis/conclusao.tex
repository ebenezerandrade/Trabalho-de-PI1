\chapter[Conclusões]{Conclusões}

	Com a fase inicial do projeto foi possível estipular as condições iniciais do Parque Urbano do Gama-DF, além de estabelecer ideias que possam enriquecer o projeto. Foi observado que nem todos os problemas poderão ser abordados, uma vez que o local de referência tem uma legislação de proteção ambiental que limita o escopo do projeto. Além disso, o tempo de execução do projeto não é extenso. Problemas de comunicação entre o grupo foram encontrados e muitos deles solucionados. No entanto, o problema mais frequente encontrado no desenvolvimento do projeto até então tem sido concluir as tarefas no tempo estipulado, devido a atrasos individuais e inter dependência entre os subsistemas do projeto. No ponto de controle 2 foi possível obter com maior clareza detalhamentos técnicos das soluções apresentadas. Pelo projeto ser extenso, muitos entregáveis são abordados e a profundidade no detalhamento de cada uma das soluções ainda se mostra superficial, esperando que isso possa ser modificado para ponto de controle 3, viabilizando um projeto bem estruturado e com custos apresentados. 