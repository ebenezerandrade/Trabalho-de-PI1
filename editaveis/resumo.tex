\begin{resumo}
O Projeto desenvolvido \'e constitu\'ido por um conjunto de alunos de Engenharia com o aux\'ilio de orientadores que visam elaborar um conjunto de pesquisas e planejamentos que ir\~ao resultar em um projeto de melhorias que utilizam exclusivamente solu\c{c}\~oes tecnol\'ogicas para o Parque Urbano do Gama-DF. Esse deve ser detalhado de modo a ser poss\'ivel uma futura utiliza\c{c}\~ao de ideias descritas atrav\'es de simula\c{c}\~oes, esquem\'aticos e texto descritivo. A metodologia adotada foi o \textit{SCRUM}. Inicialmente, uma pesquisa sobre o parque foi realizada atrav\'es de visita t\'ecnica, visita \`a Administra\c{c}\~ao do Gama e pesquisa por outras fontes de informa\c{c}\~ao.  Um contato inicial foi realizado com moradores dos arredores visando determinar os principais problemas a serem enfrentados durante o projeto. Com base nisso uma EAP (Estrutura Anal\'itica de Projeto) e um Escopo foram determinados. Poss\'iveis solu\c{c}\~oes e estudos de viabilidade foram efetuados e ser\~ao submetidos \`a aprova\c{c}\~ao por uma banca de avaliadores no primeiro ponto de controle, visando o controle de qualidade do projeto executado.

 \vspace{\onelineskip}
    
 \noindent
 \textbf{Palavras-chaves}: Parque. Gama. Revitaliza\c{c}\~ao
\end{resumo}
